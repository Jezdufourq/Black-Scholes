\documentclass[12pt]{article}
\usepackage{lipsum}
\usepackage{amsmath, amssymb}
\usepackage[margin=2cm]{geometry} % Margin
\usepackage{float} % Allows you to control float positions
\usepackage{graphicx} % Allows you to import images
\usepackage{multirow}
\usepackage[myheadings]{fullpage}
\usepackage{fancyhdr}
\usepackage{lastpage}
\usepackage{array}
\usepackage{amsmath}
% !TeX spellcheck = en_US
\usepackage{listings}
\usepackage{xcolor} % for setting colors
\usepackage{cite}
\usepackage{algorithm}
\usepackage{algorithmic}

\newcommand{\HRule}[1]{\rule{\linewidth}{#1}}
\setcounter{tocdepth}{5}
\setcounter{secnumdepth}{5}
% set the default code style
\lstset{
	frame=tb, % draw a frame at the top and bottom of the code block
	tabsize=4, % tab space width
	showstringspaces=false, % don't mark spaces in strings
	numbers=left, % display line numbers on the left
	commentstyle=\color{green}, % comment color
	keywordstyle=\color{blue}, % keyword color
	stringstyle=\color{red}, % string color
	breaklines=true, % wrapping the lines back around
	postbreak=\mbox{\textcolor{red}{$\hookrightarrow$}\space} % Annotating the wrapped lines
}

\newcolumntype{L}[1]{>{\raggedright\let\newline\\\arraybackslash\hspace{0pt}}m{#1}}

%-------------------------------------------------------------------------------
% HEADER & FOOTER
%-------------------------------------------------------------------------------
\usepackage{fancyhdr}
\setlength{\headheight}{15.2pt}
\pagestyle{fancy}


\author{Jeremiah Dufourq}
\begin{document}
	\begin{titlepage}
		\centering
		
		
		{\scshape\LARGE Queensland University of Technology \par}
		\vspace{5cm}
		{\scshape\Large Black Scholes Model\par}
		\vspace{1.5cm}
		{\huge\bfseries Extracted from \textit{C++ for Financial Engineers}\par}
		\vspace{2cm}
		{\Large\itshape Jeremiah Dufourq\par}
		
		% Bottom of the page
		{\large \today\par}
	\end{titlepage}
	
	
	\newpage
	\tableofcontents
	\newpage
	
	
	\section{Rationale:}
	The purpose of this document to to gain an insight into how the Black Scholes Model functions, and how a C++ model can be developed to price call and put options using the Black Scholes Method. 
	
	To give some context, this report will be adapted from the works of Daniel J. Duffy, in his book \textit{Introduction to C++ for Financial Engineers}. The goal of this book was to introduce the reader to the C++ programming language and its applications to the field of quantitative finance. There are three main parts to the book, which are detailed below:
	
	\begin{enumerate}
		\item C++ syntax 
		\item C++ design patterns, data structures, and libraries
		\item C++ quantitative finance applications
	\end{enumerate}
	
	\section{Introduction}
	
	\section{Black Scholes}
	
	We can derive the Black Scholes PDE using the delta hedging argument.
	
	\begin{equation}
	\frac{\partial V}{\partial t} + \frac{1}{2}\sigma^2S^2 \frac{\partial^2 V}{\partial S^2}+rS\frac{\partial V}{\partial S} - rV = 0
	\end{equation}
	
	The key assumptions of this model are as follows:
	
	The price of a stock follows a geometric brownian motion
	\begin{equation}
	dS_{t}= \mu S_{t} dt + \sigma S_{t} dw_{t}
	\end{equation}
	
	Whereby, $\mu$ $\sigma$ are constants
	
	The price of an option is a function of the following:
	
	
	$V = V(T-t, S_{t}; r, \sigma, k)$
	Whereby,
	\begin{align*}
	T - t &= Time to maturity of option \\
	S_{t} &= Stock price \\
	r &= Risk free rate \\
	\sigma &= Volatility \\
	K &= Strike price on option \\
	\end{align*}
	
	Given that $\sigma, r, K$ are constant, we can then rewrite;
	
	\begin{equation}
	V = V(T-t,S_{t}) \iff V_{t}
	\end{equation}
	
	The value of a bank account amount has no stochastic property, and can therefore be written as:
	
	\begin{equation}
	dB = rBdt
	\end{equation}
	Whereby,
	
	\begin{align*}
	r &= risk free rate. \\
	B &= Bank account. \\
	\end{align*}
	
	Using the above assumptions, we can derive the Black Scholes PDE equation.
	
	We know through Ito's Lemma, the follow is true;
	
	\begin{equation}
	dV = \frac{\partial V}{\partial t}dt + \frac{\partial V}{\partial S}dS + \frac{1}{2}\frac{\partial^2 V}{\partial s^2}dS^2
	\end{equation}
	This asserts that the derivative of a stochastic function is equal to the above formula.
	We can use the above to substitute into Ito's Lemma
	Taking the 2nd order differential of $dS$ with respect to time;
	
	\begin{align*}
	\frac{dS^2}{dt} &= \frac{dS}{dt}[\mu Sdt] + \frac{dS}{dt}[\sigma SdW]
	\end{align*}
	
	we can then use the product rule,
	\begin{align*}
	\frac{dS}{dt} &= \sigma ^2 S^2 dt 
	\end{align*}
	
	We ca then substitute into Ito's Lemma to get;
	
	
	
	
	\lhead{Jeremiah Dufourq}
	\rhead{Black Scholes Model}
	
	
	
	
	
	
	
	
	
	
	
	
	
\end{document}          